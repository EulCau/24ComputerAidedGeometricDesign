\documentclass{article}
\usepackage{ctex}
\usepackage{amsmath}
\usepackage{amssymb}

\setlength{\parindent}{2em}

\begin{document}
    \title{计算机辅助几何设计作业三}
    \author{姓名: 刘行 \quad 学号: PB22000150}
    \date{}
    \maketitle

    \section*{第1题: 证明 Bézier 曲线的弧长不大于控制多边形的周长}
        设 Bézier 曲线为
        \begin{equation*}
            \mathbf{B}\left(t\right) = \sum_{i=0}^{n} B_{i} \, b_{i,n}\left(t\right), \quad t \in [0,1], \quad b_{i,n}\left(t\right) = \binom{n}{i} t^{i} (1-t)^{n-i}
        \end{equation*}

        弧长公式为:
        \begin{equation*}
            L = \int_{0}^{1} \left\lvert\mathbf{B}^{\prime}(t)\right\rvert \, \text{d}t.
        \end{equation*}

        而 Bézier 曲线的导数为:
        \begin{equation*}
            \mathbf{B}^{\prime}(t) = n \sum_{i=0}^{n-1} (B_{i+1} - B_{i}) \, b_{i,n-1}\left(t\right).
        \end{equation*}

        由三角不等式:
        \begin{equation*}
            \left\lvert\mathbf{B}^{\prime}\left(t\right)\right\rvert \leq n \sum_{i=0}^{n-1} \left\lvert B_{i+1}-B_{i}\right\rvert \, b_{i,n-1}\left(t\right).
        \end{equation*}

        两边对 $t$ 从 0 到 1 积分:
        \begin{equation*}
            L \leq n \sum_{i=0}^{n-1} \left\lvert B_{i+1}-B_{i}\right\rvert \int_{0}^{1} b_{i,n-1}\left(t\right) \, \text{d}t.
        \end{equation*}

        由 $\int_{0}^{1} b_{i,n-1}\left(t\right) \, \text{d}t = \frac{1}{n}$, 得:
        \begin{equation*}
            L \leq \sum_{i=0}^{n-1} \left\lvert B_{i+1}-B_{i}\right\rvert.
        \end{equation*}

        右边即控制多边形的周长, 故证得:
        \begin{equation*}
            L_{\text{Bézier}} \leq L_{\text{control polygon}}.
        \end{equation*}

    \section*{第2题: 证明圆弧不能被 Bézier 曲线精确表示}
        若圆弧能被 Bézier 曲线精确表示, 则 Bézier 曲线应满足圆的方程:
        \begin{equation*}
            x^{2} + y^{2} = R^{2}.
        \end{equation*}

        然而, $x\left(t\right)$ 和 $y\left(t\right)$ 都是关于 $t$ 的多项式, 因此 $x^{2} + y^{2}$ 也是多项式, 且因为 $x^{2}$ 与 $y^{2}$ 最高次项系数均为正, 它们的和的最高次项系数也为正. 于是 $x^{2}$ 与 $y^{2}$ 最高次项均为零次. 曲线退化为点. 矛盾.

    \section*{第3题: 求平面 n 次 Bézier 曲线与原点及首末控制点围成区域的面积}
        设 Bézier 曲线为:
        \begin{equation*}
            \mathbf{B}\left(t\right) = \left(x\left(t\right), y\left(t\right)\right) = \sum_{i=0}^{n} \left(x_{i}, y_{i}\right) \, b_{i,n}(t).
        \end{equation*}

        由格林公式, 曲线与两直线 (连接原点与端点) 围成区域的有向面积为:
        \begin{equation*}
            A = \frac{1}{2}\int_{0}^{1} \left( x\left(t\right)y^{\prime}\left(t\right) - y\left(t\right)x^{\prime}\left(t\right) \right) \, \text{d}t.
        \end{equation*}

        代入 Bézier 形式:
        \begin{equation*}
            x^{\prime}\left(t\right) = n \sum_{i=0}^{n-1} (x_{i+1}-x_{i}) b_{i,n-1}\left(t\right), \quad
            y^{\prime}\left(t\right) = n \sum_{i=0}^{n-1} (y_{i+1}-y_{i}) b_{i,n-1}\left(t\right).
        \end{equation*}

        于是:
        \begin{equation*}
            A = \frac{n}{2}\int_{0}^{1} \sum_{i=0}^{n}\sum_{j=0}^{n-1} \left[x_{i}\left(y_{j+1}-y_{j}\right) - y_{i}(x_{j+1}-x_{j})\right] \, b_{i,n}\left(t\right) \, b_{j,n-1}\left(t\right) \, \text{d}t.
        \end{equation*}

        由 Bernstein 基函数的积积分恒等式:
        \begin{equation*}
            \int_{0}^{1} b_{i,n}\left(t\right) b_{j,n-1}\left(t\right) \, \text{d}t = \frac{1}{\left(2n\right)!} \cdot \frac{\left(n+i\right)! \, \left(n-j-1\right)!}{i! \, j!}.
        \end{equation*}

        因此:
        \begin{equation*}
        A = \frac{n}{2} \sum_{i=0}^{n}\sum_{j=0}^{n-1} \left[x_{i}\left(y_{j+1} - y_{j}\right) - y_{i}\left(x_{j+1} - x_{j}\right)\right] \int_{0}^{1} b_{i,n}\left(t\right) \, b_{j,n-1}(t) \, \text{d}t.
        \end{equation*}

        这就是面积的显式公式, 它完全由控制点 $(x_i,y_i)$ 表达.
\end{document}
