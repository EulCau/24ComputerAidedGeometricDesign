% homework4_problem1.tex
\documentclass[12pt]{ctexart}
\usepackage{amsmath,amssymb,amsthm}
\setlength{\parindent}{2em} % 首行缩进两字符
\allowdisplaybreaks

\title{计算机辅助几何设计 — 作业 4\\题目 1(证明)}
\author{}
\date{2025 年 10 月 13 日}

\begin{document}
\maketitle

\noindent\textbf{题目:} 设 $f(x)\in C^2[a,b]$ 为任意插值函数,$S(x)$ 为自然三次样条插值函数(两端点处二阶导数为零),证明
\[
\int_a^b \bigl(S''(x)\bigr)^2\,dx \le \int_a^b \bigl(f''(x)\bigr)^2\,dx,
\]
并且当且仅当 $f(x)\equiv S(x)$ 时取等号。

\bigskip

\noindent\textbf{证明:}

\noindent 设插值节点为
\[
a=x_0<x_1<\cdots<x_n=b,
\]
并且规定 $f(x_i)=S(x_i)=y_i$,对任意满足相同插值条件的 $C^2$ 函数 $f$ 定义误差函数
\[
\eta(x)=f(x)-S(x).
\]
由插值条件有 $\eta(x_i)=0$,$i=0,1,\dots,n$。

\medskip

\noindent 考察积分
\[
\int_a^b \bigl(f''(x)\bigr)^2\,dx
= \int_a^b \bigl(S''(x)+\eta''(x)\bigr)^2\,dx
= \int_a^b \bigl(S''(x)\bigr)^2\,dx
  + 2\int_a^b S''(x)\eta''(x)\,dx
  + \int_a^b \bigl(\eta''(x)\bigr)^2\,dx.
\]

\noindent 我们的关键目标是证明交叉项为零,即
\[
\int_a^b S''(x)\eta''(x)\,dx = 0.
\]

\noindent 由自然三次样条的边界条件知 $S''(a)=S''(b)=0$。对区间 $[a,b]$ 分段积分(在每个子区间 $[x_i,x_{i+1}]$ 上分别处理)并利用样条在每一小区间上为三次多项式的事实,可以得到交叉项为零。下面给出一个较为直接的分部积分证明。

\medskip

\noindent 首先分部积分一次:
\[
\int_a^b S''(x)\eta''(x)\,dx
= \bigl[S''(x)\eta'(x)\bigr]_a^b - \int_a^b S'''(x)\eta'(x)\,dx.
\]
由于 $S''(a)=S''(b)=0$,第一项为零,因此
\[
\int_a^b S''(x)\eta''(x)\,dx = -\int_a^b S'''(x)\eta'(x)\,dx.
\]

\noindent 接下来在每个子区间 $[x_i,x_{i+1}]$ 上再分部积分一次(注意 $S'''(x)$ 在每个子区间内为常数,因为 $S$ 在每段上是三次多项式):
\[
\int_{x_i}^{x_{i+1}} S'''(x)\eta'(x)\,dx
= \bigl[S'''(x)\eta(x)\bigr]_{x_i}^{x_{i+1}} - \int_{x_i}^{x_{i+1}} S^{(4)}(x)\eta(x)\,dx.
\]
但对三次多项式有 $S^{(4)}(x)=0$,于是
\[
\int_{x_i}^{x_{i+1}} S'''(x)\eta'(x)\,dx = S'''(x)\bigl(\eta(x_{i+1})-\eta(x_i)\bigr).
\]
由于对所有节点 $\eta(x_i)=0$,右端为零。对所有区间求和得到
\[
\int_a^b S'''(x)\eta'(x)\,dx = 0.
\]
因此有
\[
\int_a^b S''(x)\eta''(x)\,dx = 0.
\]

\medskip

\noindent 将该结果代回前式,可得
\[
\int_a^b \bigl(f''(x)\bigr)^2\,dx
= \int_a^b \bigl(S''(x)\bigr)^2\,dx + \int_a^b \bigl(\eta''(x)\bigr)^2\,dx
\ge \int_a^b \bigl(S''(x)\bigr)^2\,dx.
\]
由此得到所需不等式。并且等号成立当且仅当 $\int_a^b (\eta''(x))^2\,dx=0$,即当且仅当 $\eta''(x)\equiv 0$。因为 $\eta$ 在所有节点处为 0,且 $\eta''\equiv 0$ 意味着 $\eta$ 在每段上为一次多项式,但若一个一次多项式在至少两个不同点为零,则该多项式恒为零,于是 $\eta\equiv 0$,即 $f\equiv S$。

\qed

\end{document}
